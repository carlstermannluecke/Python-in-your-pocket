\documentclass[10pt,a4paper]{article}
\usepackage[utf8]{inputenc}
\usepackage{amsmath}
\usepackage{amsfonts}
\usepackage{amssymb}
\usepackage[left=2cm,right=2cm,top=2cm,bottom=2cm]{geometry}
%\usepackage{natbib}
%\usepackage{hyperref}
\usepackage[colorlinks = true,
            linkcolor = blue,
            urlcolor  = blue,
            citecolor = blue,
            anchorcolor = blue]{hyperref}
\author{Carl Stermann-L\"ucke, and Zainab Nazari}
\title{Python in your pocket}
\begin{document}
\maketitle
\tableofcontents
\section{Installation}
If you own a Google account you do not need to install anything! This is the easiest that you can reach to a python notebook.
In the website:https://colab.research.google.com you can read and write and run the python codes.
We generally need to tell apart two different ways of running python:
\begin{itemize}
\item scripts
\item notebooks
\end{itemize}
\subsection{Mac}
\section{Basics}
\subsection{Operators}
In python we have all the basic mathemetical operators. This includes...
\subsection{Indentation}
with the Hamiltonian $H(x,p)$ encoding the total energy of the system.
\subsection{Loops}
\input{python_training/loop-explanation1.tex}
I am going to make it work. \cite{al} Here is the eq
\begin{equation}\label{eq1}
x=y^2
\end{equation}

%\begin{figure}
%\includegraphics[width=1.08 \textwidth]{s6}
%\end{figure}

Now I would like to cite it (\ref{eq1})
\bibliographystyle{plain}
\bibliography{refe}
\end{document}